\documentclass[a4paper]{article}

\usepackage[english]{babel}
\usepackage[utf8]{inputenc}
\usepackage{amsmath}
\usepackage{graphicx}
\usepackage[colorinlistoftodos]{todonotes}

\title{Building a Student Intervention System}

\author{Uirá Caiado}

\date{\today}

\begin{document}
\maketitle

\begin{abstract}
There exists a push from educators and administrators to raise the likelihood of students successfully complete their programs. The aim of this project is to identify students who need intervention before they drop out of school. My goal is to use concepts from supervised machine learning to find the most effective model with the least amount of computation costs that identify such students.
\end{abstract}

\section{Introduction}
\label{sec:introduction}

\subsection{Some Background}
As stated by \cite{Udacity} in the description for this project, as education has grown to rely more and more on technology, more and more data is available for examination and prediction. Logs of student activities, grades, interactions with teachers and fellow students, and more are now captured through learning management systems.

Within all levels of education, there exists a surge to help increase the likelihood of student success without watering down the education or engaging in behaviors that raise the probability of passing metrics without improving the actual underlying learning. Graduation rates are often the criteria of choice for this, and educators and administrators are after new ways to predict success and failure early enough to stage effective interventions, as well as to identify the effectiveness of different interventions.

\subsection{The Goal}
The goal for this project is implementing a student intervention system using concepts from Supervised Machine Learning. I am going to choose and develop a model that will predict the likelihood that a given student will pass, thus helping diagnose whether or not an intervention is necessary.

I will suppose that the data available to create the model is a representative but a small sample of the dataset that I would have access in the production environment. So, besides the typical accuracy requirements for any machine-learning project, I am also going to look for models that are efficient in the use of resources (computation time and memory). Thus, the model will be evaluated on three factors:

\begin{itemize}
\item Its F1 Score, summarizing the number of correct positives and correct negatives out of all possible cases. In other words, how well does the model differentiate likely passes from failures
\item The size of the training set, preferring smaller training sets over larger ones. That is, how much data does the model need to make a reasonable prediction?
\item The computation resources to make a reliable prediction. How much time and memory is required to correctly identify students that need intervention?
\end{itemize}

\subsection{Classification vs Regression}
The model that will be developed is a Classifier. According to \cite{Hastie_2009}, the distinction in the output type has led to a naming convention for prediction tasks: \textit{regression} when we predict quantitative (continuous) outputs, and \textit{classification} when we predict qualitative outputs (discrete). As the goal of this project is to identify if the student will succeed, the problem posed is a classification problem once it requires a binary answer (passed, failed).

\section{Theory 2-3 pages}
\label{sec:theory}

\subsection{Two-dimensional Electron Gas}
Here, explain the concept of a 2-DEG in GaAs/AlGaAs. What is a 2-DEG and why does it arise?

\subsection{Hall Effect}
Explain the classical Hall effect in your own words. What do I measure at $B=0$? And what happens if $B>0$? Which effect gives rise to the voltage drop in the vertical direction?

\subsection{Quantum Hall Effect}
Explain the IQHE in your own words. What does the density of states look like in a 2-DEG when $B=0$? What are Landau levels and how do they arise? What are edge states? What does the electron transport look like when you change the magnetic field? What do you expect to measure?

\section{Experiment 1-2 pages}
\subsection{Fabrication}
Explain a step-by-step recipe for fabrication here. How long did you etch and why? What is an Ohmic contact?
\subsection{Experimental set-up}
Explain the experimental set-up here. Use a schematic picture (make it yourself in photoshop, paint, ...) to show how the components are connected. Briefly explain how a lock-in amplifier works.

\section{Results and interpretation 2-3 pages}
Show a graph of the longitudinal resistivity ($\rho_{xx}$) and Hall resistivity ($\rho_{xy}$) versus magnetic field, extracted from the raw data shown in figure \ref{fig:data}. You will have the link to the data in your absalon messages, if not e-mail Guen (guen@nbi.dk). Explain how you calculated these values, and refer to the theory.

\begin{figure}
\centering
\includegraphics[width=1\textwidth]{raw_data.png}
\caption{\label{fig:data}Raw (unprocessed) data. Replace this figure with the one you've made, that shows the resistivity.}
\end{figure}

\subsection{Classical regime}
Calculate the sheet electron density $n_{s}$ and electron mobility $\mu$ from the data in the low-field regime, and refer to the theory in section \ref{sec:theory}. Explain how you retrieved the values from the data (did you use a linear fit?).
Round values off to 1 or 2 significant digits: 8.1643 ~= 8.2. Also, 5e-6 is easier to read than 0.000005.

!OBS: This part is optional (only if you have time left).
Calculate the uncertainty as follows: \newline $u(f(x, y, z)) = \sqrt{(\frac{\delta f}{\delta{x}} u(x))^{2} + (\frac{\delta f}{\delta{y}} u(y))^{2} + (\frac{\delta f}{\delta{z}} u(z))^{2}}$, where $f$ is the calculated value ($n_{s}$ or $\mu$), $x, y, z$ are the variables taken from the measurement and $u(x)$ is the uncertainty in x (and so on).

\subsection{Quantum regime}
Calculate $n_{s}$ for the high-field regime.
Show a graph of the longitudinal conductivity ($\rho_{xx}$) and Hall conductivity($\rho_{xy}$) \textbf{in units of the resistance quantum} ($\frac{h}{e^{2}}$), depicting the integer filling factors for each plateau.
Show a graph of the plateau number versus its corresponding value of $1/B$. From this you can determine the slope, which you use to calculate the electron density.
Again, calculate the uncertainty for your obtained values.

\section{Discussion 1/2-1 page}
Discuss your results. Compare the two values of $n_{s}$ that you've found in the previous section. Compare your results with literature and comment on the difference. If you didn't know the value of the resistance quantum, would you be able to deduce it from your measurements? If yes/no, why?

\newpage
\section{Some LaTeX tips}
\label{sec:latex}
\subsection{How to Include Figures}

First you have to upload the image file (JPEG, PNG or PDF) from your computer to writeLaTeX using the upload link the project menu. Then use the includegraphics command to include it in your document. Use the figure environment and the caption command to add a number and a caption to your figure. See the code for Figure \ref{fig:frog} in this section for an example.

\begin{figure}
\centering
\includegraphics[width=0.3\textwidth]{frog.jpg}
\caption{\label{fig:frog}This frog was uploaded to writeLaTeX via the project menu.}
\end{figure}

\subsection{How to Make Tables}

Use the table and tabular commands for basic tables --- see Table~\ref{tab:widgets}, for example.

\begin{table}
\centering
\begin{tabular}{l|r}
Item & Quantity \\\hline
Widgets & 42 \\
Gadgets & 13
\end{tabular}
\caption{\label{tab:widgets}An example table.}
\end{table}

\subsection{How to Write Mathematics}

\LaTeX{} is great at typesetting mathematics. Let $X_1, X_2, \ldots, X_n$ be a sequence of independent and identically distributed random variables with $\text{E}[X_i] = \mu$ and $\text{Var}[X_i] = \sigma^2 < \infty$, and let

\begin{equation}
S_n = \frac{X_1 + X_2 + \cdots + X_n}{n}
      = \frac{1}{n}\sum_{i}^{n} X_i
\label{eq:sn}
\end{equation}

denote their mean. Then as $n$ approaches infinity, the random variables $\sqrt{n}(S_n - \mu)$ converge in distribution to a normal $\mathcal{N}(0, \sigma^2)$.

The equation \ref{eq:sn} is very nice.

\subsection{How to Make Sections and Subsections}

Use section and subsection commands to organize your document. \LaTeX{} handles all the formatting and numbering automatically. Use ref and label commands for cross-references.

\subsection{How to Make Lists}

You can make lists with automatic numbering \dots

\begin{enumerate}
\item Like this,
\item and like this.
\end{enumerate}
\dots or bullet points \dots
\begin{itemize}
\item Like this,
\item and like this.
\end{itemize}
\dots or with words and descriptions \dots
\begin{description}
\item[Word] Definition
\item[Concept] Explanation
\item[Idea] Text
\end{description}

We hope you find write\LaTeX\ useful, and please let us know if you have any feedback using the help menu above.

\begin{thebibliography}{9}
\bibitem{nano3}
  K. Grove-Rasmussen og Jesper Nygård,
  \emph{Kvantefænomener i Nanosystemer}.
  Niels Bohr Institute \& Nano-Science Center, Københavns Universitet

\end{thebibliography}
\end{document}